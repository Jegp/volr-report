\documentclass[a4paper,oneside]{memoir}
\usepackage[english]{babel}
\usepackage[T1]{fontenc}
\usepackage[utf8]{inputenc}
\usepackage{wallpaper}
\usepackage{palatino}
\usepackage{hyperref}
\usepackage{csquotes}
%linespacing
\usepackage{setspace}
\renewcommand{\baselinestretch}{1.5}

% Bibliography
\usepackage[style=authoryear]{biblatex}
\addbibresource{bib.bib}
\bibliography{bib}

% Promote sections and subsections
\setheadfoot{\onelineskip}{2\onelineskip}
\setheaderspaces{*}{1mm}{*}
% \chapterstyle{plain} % needed?
\checkandfixthelayout

\renewcommand{\thesection}{\arabic{section}}
\makeatletter
\let\l@section\l@chapter
\makeatother

\renewcommand{\thesection}{\arabic{section}}
\renewcommand{\thesubsection}{\thesection.\arabic{subsection}}
\makeatletter
\let\l@subsection\l@section
\let\l@section\l@chapter
\makeatother

% Glossary
\usepackage[numberedsection=nameref]{glossaries}
\renewcommand{\glossarypreamble}{\label{glos}}
\makeglossaries
\newglossaryentry{ai} {
    name = artificial intelligence,
    description = {Artificial intelligence (AI) covers the broad discipline in computer science
that is concerned with replicating intelligent behaviour in computational systems. The exact
definition is controversial for historical reasons \autocite{Nilsson2009}}
}
\newglossaryentry{computation} {
   name = computation,
   description = {Computation refers to any process (in any
substrate) that can deduce new information based on old information. In
this is manifested as computing instructions}
}
\newglossaryentry{dsl} {
  name = domain specific language,
  description = {A DSL is a language used to model concepts from a certain
    domain. DSLs are usually simpler than more general programming languages in
    that they contain fewer concepts and less complex syntax}
}
\newglossaryentry{futhark} {
   name = {Futhark},
   description = {A programming language geared towards performance in parallel environment such as
   graphics processors (GPUs). Futhark is a purely functional array language and is
   developed by HIPERFIT research center under the Department of Computer Science at the
   University of Copenhagen (DIKU)}
}
\newglossaryentry{ncc} {
   name = {NCC},
   description = {Neural patterns or condition that is minimally sufficient for a conscious
thought to occur. See \autocite{atkinson2000, Hohwy2009}}
}
\newglossaryentry{ml} {
  name = machine learning,
  description = {Machine learning is a sub-field within \gls{ai} that is concerned
    with developing systems that "progressively improves their performance on a
    certain task" \autocite{wiki:ml}}
}
\newglossaryentry{meme} {
name = meme,
description = {\textit{Meme} is a shortened form of the ancient Greek \textit{mimeme} meaning
'imitated thing' and was coined by Richard Dawkins. A meme refers to a idea or a
\textit{way of behaving} that can be \enquote{copied, transmitted, remembered, taught, shunned,
brandished, ridiculed, parodied, censored, hallowed} \autocite{dennett2017}}
}
\newglossaryentry{opencl} {
   name = {OpenCL},
   description = {An open standard for cross-platform parallel programming, which
   allows software to be executed on CPUs, GPUs or other processors or hardware accelerators. See \url{
   https://www.khronos.org/opencl/}}
}
\newglossaryentry{ref} {
  name = REF,
  description = {A model for rehabilitation in patients with brain lesions, developed
    by \cite{Mogensen2011}. An extension in the form of the REFGEN model was developed by
    \cite{Mogensen2017}}
}

\makeglossaries
% Setup captions
\captionstyle[\centering]{\centering}
\changecaptionwidth
\captionwidth{0.8\linewidth}

% Protect against widows and orphans
%\clubpenalty=10000
%\widowpenalty=10000

%\linespread{1.2}

\raggedbottom

\chapterstyle{ger}

\maxsecnumdepth{subsection}

%%  Setup fancy style quotation
%%  ==================================================================
%\usepackage{tikz}
%\usepackage{framed}

%\newcommand*\quotefont{\fontfamily{fxl}} % selects Libertine for quote font

% Make commands for the quotes
%\newcommand*{\openquote}{\tikz[remember picture,overlay,xshift=-15pt,yshift=-10pt]
%     \node (OQ) {\quotefont\fontsize{60}{60}\selectfont``};\kern0pt}
%\newcommand*{\closequote}{\tikz[remember picture,overlay,xshift=15pt,yshift=5pt]
%     \node (CQ) {\quotefont\fontsize{60}{60}\selectfont''};}

% select a colour for the shading
%\definecolor{shadecolor}{rgb}{1,1,1}

% wrap everything in its own environment
%\newenvironment{shadequote}%
%{\begin{snugshade}\begin{quote}\openquote}
%{\hfill\closequote\end{quote}\end{snugshade}}

%%  Begin document
%%  ==================================================================
\begin{document}

%%  Begin title page
%%  ==================================================================
    \thispagestyle{empty}
    \ULCornerWallPaper{1}{ku-coverpage/nat-farve.pdf}
    \ULCornerWallPaper{1}{ku-coverpage/diku-en.pdf}
    \begin{adjustwidth}{-3cm}{-1.5cm}
    %\vspace*{-1cm}
    %\textbf{\Huge Free topic} \\
    \vspace*{2.5cm}
    \textbf{\Huge Modelling learning systems} \\
    \vspace*{.1cm}
    {\Huge  A DSL for cognitive neuroscientist}\\
    \begin{tabbing}
    % adjust the hspace below for the longest author name
    Jens Egholm Pedersen \hspace{1cm} \= \texttt{<xtp778@alumni.ku.dk>} \\
    \\[11cm]

    \textbf{\Large Supervisor} \\
    Martin Elsman \hspace{1cm} \texttt{<mael@di.ku.dk>}
    \end{tabbing}
    \end{adjustwidth}
    \newpage
    \ClearWallPaper
%%  ==================================================================
%%  End title page


\section{Introduction}
In the past years machine learning has surpassed humans in some recognition
tasks, and the development shows no signs of slowing down.
These developments are however based on relatively old research on neural
networks \autocite{Nilsson2009, russel2007}.
Newer investigation into rehabilitation and learning indicates that such
networks alone cannot account for the same amount of learning that happens
in the brain \autocite{Mogensen2011, block2007, russel2007, Moravec98, dennett2017}.
For that reason the breakthroughs in machine learning is hard to transfer
to the domain of cognitive neuroscience.

This paper takes two steps towards remedying this.
First by describing a domain-specific language (DSL) that is capable of
representing the concepts of learning systems in the domain of neuroscience.
Second by validating the DSL through the modelling of a small learning
task, which will be executed on standard machine architecture.

The hope is that the DSL will lay the foundation for a representation of
learning and learning concepts, that will serve as better models of inference
as well as accurate simulation tools for scientist.

\subsection{Structure}
This paper is structered ...

\subsection{Problem statement}
...

\section{Theory}
This section accounts for the theoretical foundation of paper and is divided
into three parts.
The first part concerns the broad topic of computation and learning in neural
systems as seen from the perspective of computational neuroscience. By focusing
on cognition, plasticity, learning and rehabilitation, it derives
the necessary and sufficient language abstractions to capture the complexity
of the domain.
The second part introduces traditional machine learning from the perspective of
computer science. These concepts will be applied when implementing learning
models in section \ref{volr}.
And lastly the theoretical background for language abstractions and domain
specific languages will be treated.

\subsection{Computation and learning in neural systems}
\begin{quote}
  Activity-dependent synaptic plasticity is widely believed to be the basic
  phenomenon underlying learning and memory \autocite{dayan2001}.
\end{quote}

Commonly referred to as \textit{what fires together, wires together}, Hebbian
learning suggests that synaptic connections from neuron $A$ to neuron $B$
are strengthened or weakened when neuron $A$ excites or inhibits the chance of
firiing neuron $B$ respectively \autocite{dayan2001}.
Hebbian learning is believed to play a large part in the plastic nature of the
brain, especially within learning and memory formation
\autocite{dayan2001, Johnston2009, Robertson1999}.

\autocite{Robertson1999} studied patients during
rehabilitation of brain damage and conjectured that learning --- whether when the
brain acquires new information or recovers from lost information --- occurs based
on the structural changes induced by the Hebbian principle
\autocite{Robertson1999}.

\autocite{Mogensen2011}

\subsubsection{Reorganisation of elementary functions}

\subsection{Machine learning}

\subsection{Language abstractions}

\section{Volr: A DSL for learning systems}
\label{volr}

\clearpage

\printglossary

\printbibliography

\end{document}
